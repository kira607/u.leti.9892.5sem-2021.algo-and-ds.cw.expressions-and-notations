\section*{Задача}

Требуется разработать класс реализующий односвязный список. 
Класс должен включать в себя следующие методы:

\begin{enumerate}
    \item \verb|push_back(data)| --- добавление в конец списка
    \item \verb|push_front(data)| --- добавление в начало списка
    \item \verb|pop_back()| --- удаление последнего элемента
    \item \verb|pop_front()| --- удаление первого элемента
    \item \verb|insert(data, index)| --- добавление элемента по индексу (вставка перед элементом, который был ранее доступен по этому индексу)
    \item \verb|at(index)| --- получение элемента по индексу
    \item \verb|remove(index)| --- удаление элемента по индексу
    \item \verb|get_size()| --- получение размера списка
    \item \verb|clear()| --- удаление всех элементов списка
    \item \verb|set(data, index)| --- замена элемента по индексу на передаваемый элемент
    \item \verb|isEmpty()| --- проверка на пустоту списка
    \item Перегрузка оператора вывода
    \item \verb|push_front(List)| --- вставка другого списка в начало
\end{enumerate}

Для выполнения задания будут реализованы два класса.

\verb|Node| будет являться узлом списка, содержать в себе данные и ссылку на следующий элемент.

\verb|List| будет непосредственным списком.
В нём будут реализованы все методы, требуемые в задании.

Класс \verb|List| содержит ссылку на первый и последний элементы списка, а так же его длину (количество элементов).

Для выполнения задания используется язык \verb|Python3| версии \verb|3.9|